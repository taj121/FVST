%!TEX root = FVST.tex
\chapter{System Examples} \label{examplesChapt}

In this chapter some sample programs are shown and the process through which they are run is explained. 

\section{Simple Swap Service}

The examples given in the paper \cite{paper1} include an example for a simple swap service which is also discussed briefly in the introduction to this paper. It consists of two clients and a coordinator. The clients send values to the coordinator who then returns to client one the value client two and vice versa.

\subsection{Input code}

\begin{minipage}{.45\textwidth}
\begin{lstlisting}[backgroundcolor=\color{white},numbers=none]
let fun coord(_) =
  let val p1 = acc-swp ()
      val x1 = recv p1
      val p2 = acc-swp ()
      val x2 = recv p2
  in send p2 x1; send p1 x2; coord ()
in spawn coord;
\end{lstlisting}
\end{minipage}
\hfill
\begin{minipage}{.45\textwidth}
\begin{lstlisting}[backgroundcolor=\color{white},numbers=none]
let fun swap(x) =
  let val p = req-swp ()
  in send p x; recv p
in spawn (fn _ => swap 1);
   spawn (fn _ => swap 2);
\end{lstlisting}
\vspace{2em}
\end{minipage} 

In this code the coordinator accepts two connects on the channel swp using acc-swp (). This gives it two endpoints, p1 and p2. After accepting each connection it receives a value over that connection. Finally the received values are sent over the alternate endpoint. 

In the swap processes a connection is requested on the swp channel using req-swp (), giving the endpoint p. A value, in this case either 1 or 2, is then sent over this endpoint and finally some other value is received over p. 

The endpoints used by the coordinator, received from calling acc-swp (), are used according the session type $?T.!T.end$. This means that on this connection a value of type $T$ is received, then a value of the same type is sent then the connection ends. 

The endpoint used by the swap function, received from the call to req-swp (), is used according to the session type $!Int.?T'.end$. In this case an int is received, a value of some type $T'$ is sent and then the connection is terminated. 

From examining these session types and the above code it is clear that types $T$ and $T'$ must be the same. This is due to the fact that the values sent by the coordinator are then received by the swap functions. Furthermore both of these types must be of the type $int$ since the values received by the coordinator are sent by the swap functions and these values are integers (1 and 2).

In this case, the programs can communicate and are typable when $T=T'=Int$. The inference algorithms proposed in (Spacasassi \& Koutavas)\cite{paper1} can infer these session types from the code given. 

\subsection{Intermediate code}

The first level of the system will produce code detailing the behaviour of the given input code and the constraints relevant to it. A simplified version of this can be seen here and the full version in the appendix. 

% \begin{figure}
Behaviours:
\begin{lstlisting}
spn (B101);
spn (psh ($l3$, S126);R131 ! int;R132 ? T125);
spn (psh ($l3$, S126);R147 ! int;R148 ? T141)
\end{lstlisting}
Constraints:
\begin{lstlisting}
Cswap1' ~ S105,
Cswap2' ~ S106,
Cswap3 ~ S126,
rec B101(psh ($l1$, S105);R114 ? T103 ;psh ($l2$, S106);R115 ? T104;R116 ! T103;R117 ! T104;B101) < B101,
$l1$ ~ R114,
$l1$ ~ R117,
$l2$ ~ R115,
$l2$ ~ R116,
$l3$ ~ R131,
$l3$ ~ R132,
$l3$ ~ R147,
$l3$ ~ R148
\end{lstlisting}
% \caption{Simplified intermediate code for Simple Swap}
% \label{simpleInter}
% \end{figure}

This code is given in the syntax developed by me for this project (see \ref{text}). In this case the first spawn of behaviour variable B101 is referencing the spawning of the coordinator in the input code. Behaviour variable B101 then references the behaviour of the coordinator. The second and third spawns reference the spawns of the swaps. 

In the constraints we can first see the three channel constraints. These are referring to the endpoints returned from the calls to acc-swp and req-swp. The next constraint is a behaviour constraint. This is what gives us the link between the behaviour variable B101 and the actual behaviour of the coordinator. Finally the last constraints are the region constraints. These either link regions to other regions or labels to regions. The labels here are static approximations of the locations of the endpoints in the code. In this case $l1$ refers to the location in the coordinator function where acc-swp is called to generate the connection for endpoint p1. $l2$ references the location of p2 and $l3$ references the call to req-swp in the spawn function. 

Since the behaviour checker does not deal with inference we must manually apply some substitutions before we can run this code through the it. This involves replacing the session variables with the actual session endpoint types and the type variables with the actual types. These substitutions are detailed in table \ref{subs}. 

\begin{table}
\centering
\begin{tabular}{l |c}
S105 & \multirow{2}{*}{?int !int end} \\
S106 & \\ \hline
S126 & !int ?int end \\ \hline
T125 & \multirow{4}{*}{int} \\
T141 & \\ 
T103 & \\
T104 & \\ 
\end{tabular}
\caption{Manual Substitutions for Simple Swap}
\label{subs}
\end{table}

\subsection{Behaviour check}

The intermediate code with substitutions can then be run through the behaviour checker detailed in \ref{checker}. Here the steps which would be taken to check the simplified code are outlined. 

First the code is run through the parser and lexical analyser. The resulting constraints are then passed into the function that deals with their internal storage. The simplified output from running this example, which includes the printing of the internal storage of the constraints, can be seen below. It can clearly be seen here how the internal storage allows for efficient look ups, particularly with respect to region constraints. The stored constraints and the behaviours are then passed to the behaviour checker. 

The first behaviour is a sequence. Once the rule is applied to deal with this the next behaviour to look at is the $spn(B101)$. At this stage the stack is empty. This behaviour matches the rule SPAWN from (fig. \ref{rules2}). This then calls the check function again with the behaviour tau and the result is anded with the result of a call to the check functions where the behaviour is the body of the spawn ($B101$). 

The rules then continue to be applied in such a fashion until a state is reached where the stack is empty, the continuation is empty and the behaviour is tau. In this case the check function returns true. Alternately if a state is reached where no rule can be applied false is returned along with a message detailing where in the rules the error state was reached. A detailed diagram of how the simplified code would be dealt with in the behaviour checker can be found in (fig. \ref{path}).

% \begin{figure}
\begin{lstlisting}
Behaviours:
Spn( B101 );
Spn( Psh( l3, ! int ? int end ); R131 ! int; R132 ? int);
Spn( Psh( l3, ! int ? int end ); R147 ! int; R148 ? int)

Paired: 
	Beta: B101 
	Behaviour: rec B101 (Psh( l1, ? int ! int end ); R114 ? int; Psh( l2, ? int ! int end ); R115 ? int; R116 ! int; R117 ! int; B101)

Region Constraints:
 label: l3
 regions:
	R132, R148, R131, R147
label: l2
 regions:
	R116, R115
label: l1
 regions:
	R117, R114

check successful!
\end{lstlisting}
% \caption{Output from Behaviour Checker for Simple Swap}
% \label{bOut}
% \end{figure}

\begin{figure}
\includegraphics[scale = 0.5]{swap1Behaviour1.png}
\caption{Simple Swap Path Taken Through Behaviour Check, Main Dig.}
\label{path}
\end{figure}

\begin{figure}
\includegraphics[scale = 0.5]{swap1Behaviour2.png}
\caption{Simple Swap Path Taken Through Behaviour Check, Sub Dig.}
\label{path2}
\end{figure}

\FloatBarrier
\section{Swap Delegation}

This example is given in the paper (Spacasassi \& Koutavas)\cite{paper1}. Here it has been altered and extended to show how incorrect code would be dealt with by the behaviour checker. 

In this example the coordinator function delegates the exchange to the clients. In the previous example the coordinator function is a bottleneck when there are large exchange values, the delegation in this example avoids this. 

\subsection{Input code}

Here the swap function connects to the coordinator using req-swp as before but now it offers two choices; SWAP and LEAD. If SWAP is selected by the coordinator then the function receives on the endpoint p then sends the value x over p. 

Otherwise if LEAD is selected the function resumes the endpoint q. This endpoint is received over p. A value is the received on q and the value passed in to the function is then sent over q. 

The coordinator accepts two connections on the swp channel and gets the two endpoints p1 and p2. It then selects SWAP on p1 and LEAD on p2. Then p1 is sent over p2 and the function recurs. 

\begin{minipage}{.45\textwidth}
\begin{lstlisting}[backgroundcolor=\color{white},numbers=none]
let fun coord(_) =
  let val p1 = acc-swp ()
  in sel-SWAP p1;
    let val p2 = acc-swp
    in sel-LEAD p2;
       deleg p2 p1;
       coord()
\end{lstlisting}
\end{minipage}
\hfill
\begin{minipage}{.45\textwidth}
\begin{lstlisting}[backgroundcolor=\color{white},numbers=none]
let fun swap(x) =
  let val p = req-swp ()
  in case p {
    SWAP: recv p; send p x
    LEAD: let val q = resume p
             val y = recv q
           in send q x; y }
\end{lstlisting}
\end{minipage}

When this coordinator function is analysed in isolation the inference algorithms will infer that the endpoints p1 and p2 have the type $\eta_{i, i \in (p1,p2)} = (\oplus \{(SWAP : \eta'), (LEAD : \eta'. end) \})$. When swap is analysed they will infer that $\eta_p = \Sigma\{(SWAP : ?T_1 .!T_2. end), (LEAD : ? \eta_q . end)\}$ and $\eta_q = ?T_2.!T_1.end$. 

When the coordinator is looked at in isolation then $\eta'$ can be any endpoint. However due to duality $\eta'$ must be equal to $\eta_q$ in this case. Also $T_1 = T_2$ must be true.

The deliberate error in this code is that in the swap function when SWAP is selected a value is received then a value is sent. When LEAD is selected the same pattern is followed. This will produce a deadlock where both instances of swap are waiting on the other to send a value. 

\subsection{Intermediate code}

A simplified version of the output from the first level of the system is given here. 

% \begin{figure}
\begin{lstlisting}
Behaviours:
spn (B118);
spn (psh ($l3$, S142);R150? optn [($SWAP$ ; R152 ! int;R151 ? T160), ($LEAD$ ; R153 ? $l4$;R154 ? T157;R155 ! int)]);
spn (psh ($l3$, S142);R181? optn [($SWAP$ ; R183 ! int;R182 ? T191), ($LEAD$ ; R184 ? $l4$ ;R185 ? T188;R186 ! int)])

\end{lstlisting}
\begin{lstlisting}
Constraints:
unit <  ses R151,
unit <  ses R182,
Cswap1' ~ S120,
Cswap2' ~ S121,
Cswap3 ~ S142,
rec B118(psh ($l1$, S120);R128 ! $SWAP$;psh ($l2$, S121);R129 ! $LEAD$;R130 ! R131;B118) < B118,
$l1$ ~ R128,
$l1$ ~ R131,
$l2$ ~ R129,
$l2$ ~ R130,
$l3$ ~ R150,
$l3$ ~ R152,
$l3$ ~ R153,
$l3$ ~ R181,
$l3$ ~ R183,
$l3$ ~ R184,
$l4$ ~ R154,
$l4$ ~ R155,
$l4$ ~ R185,
$l4$ ~ R186
\end{lstlisting}
% \caption{Simplified intermediate code for Delegated Swap }
% \label{delegInter}
% \end{figure}

Again substitutions must be made before this code can be given to the behaviour checker. These are detailed in table \ref{subs2}.

\begin{table}
\centering
\begin{tabular}{l |c}
S120 & \multirow{2}{*}{$(+) [($SWAP$; ? int ! int end ), ($LEAD$; ! ? int ! int end end)]$} \\
S121 & \\ \hline
S142 & $+ [($SWAP$; ? int ? int end), ($LEAD$; ? ? int ! int end end)] [\ ]$ \\ \hline
T160 & \multirow{4}{*}{int} \\
T191 & \\ 
T157 & \\
T188 & \\ 
\end{tabular}
\caption{Manual Substitutions for Delegated Swap}
\label{subs2}
\end{table}

\subsection{Behaviour check}

As you can see from the code given below the behaviour checker has recognised that the input code cannot communicate correctly. The error given is `$$out rule stack frame incorrect for current behaviour$$'. This tells us that the checker ran into problems when it was trying to verify a behaviour of the type $\rho!T$. We also know, from looking up the rules, that the stack frame did not match the expected form $(l; !T\eta)$. From these we can deduce that the checker ran into the error when attempting to check the swap option against the current stack frame since the option is attempting to send a value while the stack frame is expecting it to attempt to receive one 

% \begin{figure}
\begin{lstlisting}
Behaviours:
   Spn( B118 );
   Spn( Psh( l3, +[ (SWAP;? int ? int end ), (LEAD;? ? int ! int end end ) ][]) ; R150 ? optn [(SWAP; R152 ! int ; R151 ? int) (LEAD;  R153 ? l4 ;   R154 ? int ; R155 ! int) ] );  
   Spn( Psh( l3, +[ (SWAP;? int ? int end ), (LEAD;? ? int ! int end end ) ][]  ); R181 ? optn [(SWAP; R183 ! int; R182 ? int) (LEAD;  R184 ? l4 ; R185 ? int; R186 ! int) ] )
Behaviour constraints:
Beta: B118 
Behaviour: rec B118 (Psh(l1, (+)[ (SWAP;? int ! int end ), (LEAD;! ? int ! int end end )]) ; R128 ! SWAP; Psh(l2, (+)[ (SWAP;? int ! int end ), (LEAD;! ? int ! int end end )]); R129 ! LEAD; R130 ! R131;B118)

Region Constraints:
 label: l4
 regions:
R154
R185
R186
R155
label: l3
 regions:
R152
R183
R184
R181
R150
R153
label: l2
 regions:
R129
R130
label: l1
 regions:
R128
R131

Type Constraints:
Paired: 
Super: ses R182 
sub:  unit 
Paired: 
Super: ses R151
sub:  unit 

out rule
stack frame incorrect for current behaviour
ERROR: Failed Check
\end{lstlisting}
% \caption{Output from Behaviour Checker for Delegated Swap}
% \label{bOut2}
% \end{figure}

\FloatBarrier
\section{TLS}

A very simple implementation of the Transport Layer Security (TLS) handshake has been detailed here to show how it would operate under this system. Again the simplified version of the output from the two layers has been shown here and the full version has been included in the appendix. 

\subsection{Input code}

Here we see a very simple version of the TLS handshake implemented in the version of $ML_s$ accepted by the first level. In this implementation we have a client and a server. The client establishes a connection on the tls channel and sends a number, this represents the client hello message. It then receives four values on this connection. These represent the server hello,server certificate, the server exchange key and the server hello done messages. The client finally sends the client exchange key and the change cipher spec. The communication ends when the client receives the value representing the change cipher spec message from the server. 

The communications on the server side are the complements of those made by the client. Since all values are represented as integers we can conclude that the session types of these endpoints should  be (FILL ME IN) %TODO

\begin{lstlisting}
\end{lstlisting}

\subsection{Intermediate code}
Here we have taken the output from the first level and simplified it. Again the substitutions detailed in table \ref{tLSSubs} must be made before this code can be run through the behaviour checker. 

\begin{table}
\centering
\begin{tabular}{l c}
to & fill \\
\end{tabular}
\caption{TLS Substitutions}
\label{tLSSubs}
\end{table}

\begin{lstlisting}
\end{lstlisting}

\subsection{Behaviour check}

As you can see from the output below the behaviour checker shows that this communication is valid when the given substitutions are made. 

\begin{lstlisting}
\end{lstlisting}

The code given here is the simplified version. 