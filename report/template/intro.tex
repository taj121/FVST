%!TEX root = FVST.tex
\chapter{Introduction}

\section{Project Objectives}
The aims of this project are to investigate the formal verification of session types, specifically in relation to the paper Type-Based Analysis for Session Inference \cite{paper1}. Also to implement a behaviour checker based on the designs described in this paper using OCaml, Menhir and OCamllex. 

\section{Summary of Report}
% TODO
In this report first, in this section, a brief background to the area is given. The following section then gives an overview and explanation of the system proposed in the paper Type-Based Analysis for Session Inference \cite{paper1}. The third section details the implementation completed over the course of this project. The final chapters cover some detailed examples of how programs are dealt with in this system and an evaluation of the system.

\section{Motivation}
The modern world is growing increasingly dependent on distributed systems, changing the historical approach to computing dramatically. In order for modern society to function it is important that these systems communicate correctly and that when proposing or introducing new systems we can show that they will communicate correctly under all circumstances. 

Modern programming languages support data types. These allow us to use verification techniques to show that the program will run as expected on all forms of input. A similar system of types could be used for communication over distributed systems. Ideally a type system for communication would be embedded into languages in a similar fashion to the data type systems of modern languages.

This is the system that is proposed with session types. These types can specify the style of communication expected (in general terms send or receive) as well as the type of data that is expected.  

\section{Current Work}
This area is currently been researched by multiple groups. However it is currently not used in real world systems to any great extent. To date systems have been developed for applying session type disciplines to functional languages, object oriented languages and operating systems. 

\section{Background}
Traditional type systems embedded into programming languages focus on the computations and what they should produces. Session type disciplines aim for embedded session types that can describe the sequence of messages as well the type of the messages transmitted on communication channels. Then, since the session types will describe the protocol of a channel, verification techniques can be used to ensure that processes will abide by these protocols. 

\subsection{Main session type approaches}
The main approaches to session types, according to (H\"{u}ttle et al.) \cite{foundBTypes} are detailed in the following section.

\textit{Session types} are usually associated with binary communication channels where the two ends using the channel view the endpoints as complementary types. Static type checking can then be used to ensure that the communications on the channel abide by the protocol specified. 

\textit{Multiparty session types} extend binary session types to allow for more than two processes to communicate. 

\textit{Conversation session types} unify local and global multipary types and allow for an unspecified number of processes to communicate over the channel. 

\textit{Contracts} focus on general theory to confirm that communications follow the specified abstract description of input/output actions.

